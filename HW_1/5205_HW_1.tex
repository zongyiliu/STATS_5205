\documentclass[margin=1in]{article}
\usepackage[utf8]{inputenc}
\usepackage[english]{babel}
\usepackage{amsmath}
\usepackage{graphicx}
\usepackage{capt-of}

\usepackage{lipsum}
\usepackage{graphicx}
\usepackage{float}
\usepackage[margin=1in]{geometry}
\usepackage[]{amsthm} %lets us use \begin{proof}
	\usepackage[]{amssymb} %gives us the character \varnothing
	
	\title{Homework 1, STAT 5205}
	\author{Zongyi Liu}
	\date{Jan 31, 2025}
	\begin{document}
		\maketitle
		
		\subsubsection*{Question 1}
		Let $\hat{y}_i=\hat{\beta}_0+\hat{\beta}_1 x_i$ with $\hat{\beta}_0, \hat{\beta}_1$ being the least square estimators (LSE). Let $\hat{\epsilon}_i=y_i-\hat{y}_i$. Please show the two claims below.
		
		\begin{itemize}
			\item  $\sum_{i=1}^n \hat{y}_i=\sum_{i=1}^n y_i$
			\item $\sum_{i=1}^n x_i \hat{\epsilon}_i=0$
		\end{itemize}
	
	\textbf{Answer}
	
	  \underline{Property 1} For $\sum_{i=1}^n \hat{y}_i=\sum_{i=1}^n y_i$, $\hat{\epsilon_i}=y_i - \hat{y}_i$, and thus $\hat{y}_i=y_i - \hat{\epsilon_i}$. In linear regression, the residual holds a property that $\sum_{i=1}^n\hat{\epsilon_i} =0$. Thus $\sum_{i=1}^n \hat{y}_i =\sum_{i=1}^n (y_i - \hat{\epsilon_i}) $, and $\sum_{i=1}^n \hat{y}_i=\sum_{i=1}^n y_i$.
	  
	  Mathematically, it means that the regression line is designed to `balance out‘ the errors above and below the line, which leads to the sum of predicted values equaling the sum of observed values. 
	  
	  \underline{Property 2} Here we start with the definition of residuals: $ \hat{\epsilon}_i = y_i - \hat{y}_i$. Multiplying both sides by \(x_i\) and summing: $ \sum_{i=1}^{n} x_i \hat{\epsilon}_i = \sum_{i=1}^{n} x_i (y_i - \hat{y}_i)$ 
	  
	  To calculate the sum of residuals: $ \sum_{i=1}^n(\hat{\epsilon}_i)=\sum_{i=1}^n \left(y_i-\left(\beta_0+\beta_1 x_i\right)\right)$, so $\sum_{i=1}^n(\hat{\epsilon}_i)=\sum_{i=1}^n(y_i)-n \beta_0-\beta_1 \sum_{i=1}^n(x_i)$. Since the average of y (which is $\bar{y}$) is used to calculate the intercept $\left(\beta_0\right)$ and the average of x (which is $\bar{x}$) is used in the slope calculation $\left(\beta_1\right)$, the sum of residuals becomes: $\sum_{i=1}^n(\hat{\epsilon}_i)=n \bar{y}-n \beta_0-\beta_1{ }n \bar{x}=0$
	  
	  Then to prove the sum of $x$ multiplied by epsilon is zero, we have  $\sum_{i=1}^n\left(x_i \hat{\epsilon}_i\right)=\sum_{i=1}^n\left(x_i \left(y_i-\left(\beta_0+\beta_1 x_i\right)\right)\right)$. $\sum_{i=1}^n\left(x_i \hat{\epsilon}_i \right)=\sum_{i=1}^n\left(x_i y_i\right)-\beta_0 \sum_{i=1}^n(x_i)-\beta_1 \sum_{i=1}^n\left(x_i^2\right)$. Using the property that the sum of residuals is zero $(\sum_{i=1}^n \epsilon_i=0)$, and considering the average $x$ value again, this sum becomes zero: $\sum_{i=1}^n\left(x_i \hat{\epsilon}_i\right)=\sum_{i=1}^n\left(x_i y_i\right)-\beta_0 n \bar{x}-\beta_1 n \bar{x}^2=0$. Since $\frac{1}{n}\sum_{i=1}^{n} x_iy_i =\bar{xy}$, we get  $\sum_{i=1}^n\left(x_i \hat{\epsilon}_i\right) = n\bar{x} (\bar{y}-\beta_0-\beta_1{\bar{x}})=0$
	  
	  Mathematically, it means that the residuals and x values are uncorrelated, which is one of the most important premises of linear regression model.
		
		\pagebreak
		\subsubsection*{Question 2}
		Let $k_i=\frac{x_i-\bar{x}}{\sum_{i=1}^n\left(x_i-\bar{x}\right)^2}$. Please show the following properties of $k_i$ s.
		
		\begin{itemize}
		\item$\sum_{i=1}^n k_i=0$.
		\item $\sum_{i=1}^n k_i x_i=1$.
		\item $\sum_{i=1}^n k_i^2=\frac{1}{\sum_{i=1}^n\left(x_i-\bar{x}\right)^2}$.
    	\end{itemize}
    
    \textbf{Answer}
    
    
    
    \underline{For property 1}, we have $ \sum_{i=1}^n k_i = \sum_{i=1}^n \frac{x_i - \bar{x}}{\sum_{i=1}^n (x_i - \bar{x})^2}$. For  \( \bar{x} \) is the mean of \( x_1, x_2, \dots, x_n \), given by $  \bar{x} = \frac{1}{n} \sum_{i=1}^n x_i$. Since the numerator is the sum of deviations from the mean, we have $  \sum_{i=1}^n (x_i - \bar{x}) = 0$, so the total number is zero, it follows that $ \sum_{i=1}^n k_i = \frac{0}{\sum_{i=1}^n (x_i - \bar{x})^2} = 0$.
    
    \underline{Property 2}, here we have $\sum_{i=1}^n k_i x_i = \sum_{i=1}^n \frac{(x_i - \bar{x}) x_i}{\sum_{i=1}^n (x_i - \bar{x})^2}$. Rewrite \( x_i \) as \( (x_i - \bar{x}) + \bar{x} \), we get $\sum_{i=1}^n k_i x_i = \sum_{i=1}^n \frac{(x_i - \bar{x}) (\bar{x} + (x_i - \bar{x}))}{\sum_{i=1}^n (x_i - \bar{x})^2}$. Expanding the product we get $\sum_{i=1}^n k_i x_i = \sum_{i=1}^n \frac{\bar{x} (x_i - \bar{x}) + (x_i - \bar{x})^2}{\sum_{i=1}^n (x_i - \bar{x})^2}$. Then split the sum $\sum_{i=1}^n k_i x_i = \frac{\bar{x} \sum_{i=1}^n (x_i - \bar{x}) + \sum_{i=1}^n (x_i - \bar{x})^2}{\sum_{i=1}^n (x_i - \bar{x})^2}$. Since \( \sum_{i=1}^n (x_i - \bar{x}) = 0 \), the first term cancelled, and we have $
    \sum_{i=1}^n k_i x_i = \frac{\sum_{i=1}^n (x_i - \bar{x})^2}{\sum_{i=1}^n (x_i - \bar{x})^2} = 1.$ 
    
    \underline{Property 3}, we have $
    \sum_{i=1}^n k_i^2 = \sum_{i=1}^n \left( \frac{x_i - \bar{x}}{\sum_{i=1}^n (x_i - \bar{x})^2} \right)^2
    = \sum_{i=1}^n \frac{(x_i - \bar{x})^2}{\left(\sum_{i=1}^n (x_i - \bar{x})^2\right)^2}$. Then factor out the denominator $
    \sum_{i=1}^n k_i^2 = \frac{\sum_{i=1}^n (x_i - \bar{x})^2}{\left(\sum_{i=1}^n (x_i - \bar{x})^2\right)^2}
    = \frac{1}{\sum_{i=1}^n (x_i - \bar{x})^2}$. Thus we have  \( \sum_{i=1}^n k_i^2 = \frac{1}{\sum_{i=1}^n (x_i - \bar{x})^2} \).
		
		
		\pagebreak
		\subsubsection*{Question 3}
		Please watch these videos and try to understand the material.
		
		\begin{itemize}
		\item	https://www.youtube.com/watch?v=27vT-NWuw0M 
		\item https://www.youtube.com/watch?v=t-n4a18AW08
		
	\end{itemize}
		
    
	\end{document}